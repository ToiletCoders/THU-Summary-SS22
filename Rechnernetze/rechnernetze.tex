\documentclass{scrreprt}

\usepackage{graphicx}
\usepackage{tikz}
\usepackage{amstext}
\usepackage{hyperref}
\hypersetup{%
  colorlinks=false,
  linkbordercolor=blue,
  pdfborderstyle={/S/U/W 0}
}
\title{Rechnernetze}
\author{\href{https://github.com/bircni}{\color{black}github/bircni}}
\date{}
\begin{document}
\pagenumbering{gobble}  
\maketitle
\pagebreak
\renewcommand{\contentsname}{Inhaltsverzeichnis}
\newcommand\tab[1][1cm]{\hspace*{#1}}
%\renewcommand{\cftdot}{}
\setcounter{tocdepth}{1}
\tableofcontents
\addtocontents{toc}{~\hfill\textbf{Seite}\par}
\pagebreak
\pagenumbering{arabic}
\chapter{Einleitung}
\section{Datenübertragung im Internet}
\begin{itemize}
	\item Die Bitübertragungsschicht
	\begin{itemize}
		\item Bit wird in Form physikalischer Signale übertragen
		\item Übertragungsmedien
		\begin{itemize}
			\item Kupferkabel - elektrische Signale
			\item Glasfaserkabel - Lichtpulse (Itensität)
			\item Funkwellen - Amplitude, Frequenz
		\end{itemize}
		\item Problem: Übertragungsfehler wegen Signalverfälschung
	\end{itemize}
	\item Die Sicherungsschicht
	\begin{itemize}
		\item Verantwortlich für zuverlässigen Datenaustausch zwischen direkt verbundenen Rechnern
		\item Möglichkeiten: Punkt zu Punkt, Bus, Stern
		\item Aufgaben:
		\begin{itemize}
			\item Framing: Generierung der Datenpakete
			\item Fehlererkennung: Generierung der Prüfsummen
			\item (Bus)Media-Access-Control (MAC): Wer darf wann senden?
			\item (Stern)Hardware-Adressierung: Eindeutige Adressierung der Interfaces
		\end{itemize}
	\end{itemize}
	\item Die Vermittlungsschicht (IP)
	\begin{itemize}
		\item IP ist optimiert für Datenübertragung über heterogene, nicht zuverlässige Netzwerke
		\begin{itemize}
			\item Übertragung erfolgt in Form unabhängiger Pakete
			\item Einheitliches, übergreifendes Adressschema
			\item Keine Mechanismen zur Fehlerbehebung
		\end{itemize}
		\begin{figure}[h]
			\includegraphics[width=0.80\textwidth]{"graphics/IP"}
			\centering
		\end{figure}
	\end{itemize}
	\item Die Transportschicht (TCP)
	\begin{itemize}
		\item Ziel:
		\begin{itemize}
			\item Zuverlässigkeit des Datentransports
			\item Sicherung der Übertragung zwischen Anwendungsprozessen
		\end{itemize}
		\item TCP:
		\begin{itemize}
			\item Anwendung übergibt Daten an die TCP-Schicht
			\item korrekter Transport als Aufgabe von TCP
		\end{itemize}
	\end{itemize}
\end{itemize}
\section{ISO/OSI-Modell}
\begin{itemize}
	\item 7 Schichten
	\item Jede Schicht definiert Funktionen die als Dienste der nächst höheren Schicht zu Verfügung stehen
	\item keine Implementierungsvorgaben
	\item höhere Schicht nutzt die Funktionen der darunter liegenden Schicht
	\item Prinzip: "Information Hiding"
	\item Grobstruktur:
	\begin{itemize}
		\item Schicht 1-3: Netz orientiert, reine Transportfunktionalitäten, Inhalt irrelevant
		\item Schicht 4: Verbindet die Netz- und Anwendungsschicht
		\item Schicht 5-7: Anwendungs orientiert, Festlegung des Datenaustauschs und Datenformats
	\end{itemize}
	\includegraphics[width=0.4\textwidth]{"graphics/ISO"}
	\item Funktionen der Schichten:
	\begin{enumerate}
		\item Bitübertragungsschicht: (Bit-Repräsentation)
		\\ermöglicht die Übertragung unstrukturierter Bitsröme; z.B. physikalische Darstellung
		\item Sicherungsschicht: (Ethernet)
		\\dient zur Entdeckung von Übertragungsfehlern und deren Korrektur
		\item Vermittlungsschicht: (IP)
		\\ermöglicht transparente Übertragung der Daten im Netzwerk (Routing)
		\item Transportschicht: (TCP)
		\\Sicherung der Übertragung zw. zwei Anwendungen auf versch. Rechnern
		\item Sitzungsschicht: (Dialog-Steuerung)
		\\sorgt für Synchronisation und den geregelten Dialogablaug zw. zwi Anwendungsprozessen (Login)
		\item Darstellungsschicht: 
		\\Umsetzung der Darstellungen der Informationen
		\item Anwendungsschicht:
		\\einzige Zugriffsmöglichkeit der Anwendungsprozesse zur Datenübertragung (Mail,DNS)
	\end{enumerate}
	\begin{figure}[h]
		\includegraphics[width=0.50\textwidth]{"graphics/IOSTCP"}
		\centering
	\end{figure}
	\begin{figure}[h]
		\includegraphics[width=0.70\textwidth]{"graphics/ISO2"}
		\centering
	\end{figure}
\end{itemize}

\chapter{Datenübertragung}
\section{Fourieranalyse}
\label{sec:fourier}
Jede periodische Funktion g(t) mit t (Zeit) und Periode T kann als Überlagerung
von Sinus- und Cosinustermen dargestellt werden.
$$g(t)=\frac{1}{2}a_{0}+\sum_{n=1}^{\infty}{[a_{n}cos(\omega_{n}t)+b_{n}sin(\omega_{n}t)]}$$
$a_{n}$ und $b_{n}$ sind Fourierkoeffizienten mit $\omega_{n}=2\pi n/T $
\\
Der n-te Summand heißt n-te Harmonische.
\\
Ist g(t) der Spanungsverlust eines elektr. Signals dann ist $(a_{n}^{2}+b_{n}^{2})$ proportional
zur Leistung, die bei der Frequenz $f_{n}$ übertragen wird.
\\
Beispiel-Applet: \url{https://falstad.com/fourier}

\section{Dämpfung D}
\label{sec:dampf}
Üblicherweise wird die Dämpfung in der Einheit Dezibel angegeben
$$D_{dB}=10*\log_{10}(P_{in}/P_{out})  [dB]$$
$$D_{dB}=20*\log_{10}(U_{in}/U_{out})  [dB]$$
$\rightarrow$   Unabhängig davon ob Leistung [P] oder Spannung [U] verglichen werden
ergibt sich bei der Formel der gleiche Wert.
Wird als Einheit dB verwendet, addieren sich die Dämpfungen einzelner Abschnitte.

\section{Bandbreite B}
\label{sec:bandbreite}
Bandbreite eines Übertragungskanals $B = f_{max}-f_{min}$
\begin{itemize}
	\item Frequenzbereich der ohne wesentl. Dämpfung übertragen werden kann.
	\item $f_{max}$ und $f_{min}$ sind dadurch gegeben, dass die außen liegenden Frequenzen
	      unter 50\% der leistungsstärksten Frequenzen liegen.
\end{itemize}
\begin{figure}[h]
	\includegraphics[width=0.60\textwidth]{"graphics/Dampfung"}
	\centering
\end{figure}

\section{Nyquist-Theorem}
Zusammenhang zwischen \hyperref[sec:bandbreite]{Bandbreite B} und der maximal möglichen
Datenrate D eines idealen Übertragungskanals:
$$D = 2*B*\log_{2}(N)$$
$\rightarrow$ B = Bandbreite des Übertragungskanals in [Hz]
\\
$\rightarrow$ N = Anzahl der möglichen diskreten Signalstufen pro Signaländerung
\\
$\rightarrow$ D = Datenrate in bps (Bit pro Sekunde)
\\\\
Beispiel:
\begin{itemize}
	\item Binäres Signal mit N=2 und Übertragungskanal mit 3000Hz
	      $\rightarrow$ maximal erreichbare Datenrate beträgt 6000 bps
\end{itemize}

\section{Shannon'scher Kanalkapazitätssatz}
\begin{itemize}
	\item Maximale Datenrate eines realen Datenkanals
	      \begin{itemize}
		      \item D hängt vom "Signal-Rausch"-Abstand (SNR) ab
		            \\$D=B*\log_{2}(1+SNR)$
			            \\$\rightarrow$ B = Bandbreite des Übertragungskanals in [Hz]
		            \\$\rightarrow$ $SNR = P_{S}/P_{R}$
			            \\$P_{S}$ = mittlere Leistung im Nutzsignal
		            \\$P_{R}$ = mittlere Leistung im Rauschsignal
		      \item Die gebräuchliche Einheit von SNR ist [dB]
		            \\$\rightarrow$ $(SNR)_{dB} = 10*\log_{10}(SNR)$
	      \end{itemize}
	\item Beispiel
	      \begin{itemize}
		      \item Übertragungskanal mit 3000 Hz (Telefon); $(SNR)_{dB} = 30dB$
		            \\$\rightarrow$ $SNR = 1000$
			            \\$\rightarrow$ $D = 3000*\log_{2}(1+1000) \approx 30000bit/s$
	      \end{itemize}
\end{itemize}

\section{Bitrate vs. Signalgeschwindigkeit}
\begin{itemize}
	\item Signalgeschwindigkeit: Anzahl der Signalwechsel pro Sekunde
	      \begin{itemize}
		      \item Die Signalgeschwindigkeit wird in Baud [Bd] angegeben
		      \item Oft auch als "Baudrate" bezeichnet
	      \end{itemize}
	\item Bit-Rate: Anzahl der übertragenen Bits pro Sekunde
	      \begin{itemize}
		      \item Die Bitrate kann größer als die Baudrate werden
		      \item Für binäre Signalstufe (2-Stufen-Kodierung) gilt: Bitrate = Baudrate
		      \item Bei Nutzung einer 4-Stufen-Kodierung gilt: Bitrate = 2x Baudrate
	      \end{itemize}
\end{itemize}

\section{Die Ende-zu-Ende-Verzögerung von Datenpaketen}
\begin{itemize}
	\item Zeit: Datenpaketübertragung von Quell-Knoten zu Ziel-Knoten
	\item Verzögerungsarten die zur Verzögerung beitragen:
	      \\$$d_{end-to-end} = \sum_{i=1}^{N}{d^{j}_{nodal}}$$
	      \begin{itemize}
		      \item $d^{j}_{nodal}$ bezeichnet die Verzögerung in einem Knoten i
		      \item Die Knoten-Verzögerung $d^{j}_{nodal}$ setzt sich aus folgenden Anteilen zusammen:
		            \\$$d^{j}_{nodal} = d^{j}_{proc}+d^{j}_{queue}+d^{j}_{trans}+d^{j}_{prop}$$
		            \begin{itemize}
			            \item $d^{j}_{proc}$ = Verarbeitungsverzögerung (processing delay)
			            \item $d^{j}_{queue}$ = Warteschlangenverzögerung (queuing delay)
			            \item $d^{j}_{trans}$ = Übertragungsverzögerung (transmission delay)
			            \item $d^{j}_{prop}$ = Ausbreitungsverzögerung (propagation delay)
		            \end{itemize}
	      \end{itemize}
\end{itemize}
\begin{figure}[h]
	\includegraphics[width=0.80\textwidth]{"graphics/Uebertragung"}
	\centering
\end{figure}

\section{Grundlegende Übertragungstechniken}
\begin{itemize}
	\item Digitale Eingabe, digitale Übertragung: \\\hyperref[sec:Leitungscodierung]{Digitale Leitungscodierung}
	      \begin{itemize}
		      \item Beispiel: Ethernet
		            \\$\rightarrow$ Bits werden direkt als digitale Signale auf die Leitung gegeben
			            \\$\rightarrow$ Einsatz sog. Basisband-Übertragungsverfahren
	      \end{itemize}
	\item Digitale Eingabe, analoge Übertragung: \\Modulationstechniken
	      \begin{itemize}
		      \item Beispiel: DSL-Modemstrecken
		            \\$\rightarrow$ Binäre Daten werden über eine Trägerwelle übertragen
			            \\$\rightarrow$ Einsatz sog. breitband-Übertragungsverfahren
	      \end{itemize}
\end{itemize}

\section{Digitale Leitungscodierung}
\label{sec:Leitungscodierung}
\begin{itemize}
	\item Direkte Übertragung rechteckförmiger Signale
	      \begin{itemize}
		      \item Signal belegt gesamte verfügbare Bandbreite des Übertragungskanals
	      \end{itemize}
	\item Die Zuordnungsvorschrift Datenelement zwischen Signalelement heißt Signal- oder Leitungscodierung
	\item Die sich ergebende Signaverläufe heißen Signalcodes oder Übertragungscodes
	\item Erwünschte Eigenschaften von Übertragungscodes:
	      \begin{itemize}
		      \item Bittaktrückgewinnung
		      \item Codierung mehrerer Bits pro Baud (pro Signalwechsel)
		      \item Vermeidung von Gleichstromanteilen
		      \item Erkennung von Signalfehlern auf Signalebene
	      \end{itemize}
	\item Beispiele: 
		  \\
	      \includegraphics[width=0.80\textwidth]{"graphics/NRZ"}
	      \\
	      \includegraphics[width=0.80\textwidth]{"graphics/Manchester"}
	      \\
	      \includegraphics[width=0.80\textwidth]{"graphics/4B5B"}
	      \\
	      \includegraphics[width=0.70\textwidth]{"graphics/uebertragungsarten"}
	\item Synchronisation bei Bit serieller Übertragung
	      \begin{itemize}
		      \item Beispiel "RS-232-C"-Schnittstelle
		    \begin{itemize}
				\item Standart-Schnittstelle zur Übertragung alphanum. Zeichen
				\item Sender und Empfänger sind vor Datenaustausch nicht synchronisiert
				\\$\rightarrow$ Sender-/Empfängertakt müssen gleich sein
				\\$\rightarrow$ Start/Stop-Verfahren - Signalisierung von Anfang/Ende einer Übertragung
				\\$\rightarrow$ Sender-Verhalten:
				\\ Übertragung von Daten beginnt, sobald Daten anliegen, beliebige Wartezeiten
				\\$\rightarrow$ Empfänger-Verhalten:
				\\Ständige Empfangsbereitschaft
				\item Spezifikationen
				\\$\rightarrow$ "1" Signalpegel von -3V bis -15V
				\\$\rightarrow$ "0" Signalpegel von +3V bis +15V
				\\$\rightarrow$ Start-Bit setzt Leitung auf "0" und startet Taktgeber des Empfängers
				\\$\rightarrow$ Stop-Bit setzt Leitung auf "1"
			\end{itemize}
	      \end{itemize}
	\item Modulationstechniken
	\begin{itemize}
		\item Nutzung elektromag. Wellen zur Datenübertragung
		\begin{itemize}
			\item Träger wird vom Sender moduliert
			\item Empfänger demoduliert Träger und rekonstruiert Originaldaten
		\end{itemize}
		\item Amplitudendarstellung einer Trägerwelle
		$$A(t) = A_{0}*sin(2\pi ft -\phi )$$
		$A_{0}$: Amplitude; $\phi$: Phasenverschiebung;
		\\$f=1/T=$Frequenz; $T=$Schwingungsperiode;
	\end{itemize}
\end{itemize}
\chapter{Die Sicherungsschicht und lokale Netze}
Aufgaben der Sicherungsschicht:
\begin{itemize}
	\item Bereitstellung einer logischen Verbindung zwischen direkt verbundenen Kommunikationssystemen
	\item Zuverlässige Zustellung von Daten für die Vermittlungsschicht
	\\Bereitstellung einer Dienstschnittstelle
	\\Sicherung der Daten vor Verfälschung bei Übertragung
\end{itemize}
Funktion der Sicherungsschicht:
\\\includegraphics[width=0.60\textwidth]{"graphics/Funktion-Sschicht"}
\\Ablauf der Fehlererkennung:
\\\includegraphics[width=0.70\textwidth]{"graphics/Ablauf"}
\section{Rahmenbildung und Fehlererkennung}
\subsection{Rahmenbildung}
Zur Fehlererkennung werden Bitströme in kleine Dateneinheiten aufgeteilt (Rahmen).
\\Wie erkennt der Empfänger Anfang und Ende des Rahmens?
\\Das Format des Rahmens hängt von der Netztopologie ab (Ethernet, Token Ring, ATM)
\\Erkennung von Rahmengrenzen:
\begin{itemize}
	\item Verwendung illegaler Codezeichen auf Bitübertragungsbene
	\\z.B. Manchester-Codierung: kein SIgnalübergang in der Mitte eines Intervalls
	\item Längenangabe im Rahmen-Header: Byte-Zählmethode
	\item Verwendung von speziellen Steuerzeichen: Byte-Stopfen
	\\ Spezielle ASCII-Zeichen werden als Steuerzeichen benutzt (SOH,EOH)
	\\In den Daten können zufällig Steuerzeichen auftreten
\end{itemize}
\subsection{Fehlererkennung}
Aufteilung der Daten in einzelne Rahmen durch den Sender
\\Pro Rahmen wird eine redundante Zusatzinfo geschickt $\rightarrow$ Empfänger kann Übertragungsfehler erkennen
\\Hamming-Distanz:
\begin{itemize}
	\item Erlaubt die Bewertung von Fehlercodes
	\item Definition:
	\\Distanz zwischen 2 zulässigen Wörtern (Anzahl unterschiedlicher Bitpositionen)
	\\Hamming-Distanz ist die minimale Distanz zweier bel. Wörter einer Codierung
	\item Regeln:
	\\Für die Erkennung von \textit{d} Bitfehlern muss die Hamming-Distanz \textit{d+1} sein
	\\Für die Behebung von \textit{d} Bitfehlern muss die Hamming-Distanz \textit{2d+1} sein
\end{itemize}
Fehlererkennungscodes:
\begin{itemize}
	\item Eindimensionale Parität
	\\Übertragung eines zusätzlichen Bits zu jedem Wort der Länge d Bit
	\\$\rightarrow$ Ungerade Parität (d+1)tes Bit wird auf 1 gesetzt, wenn Anzahl der 1sen im d-Bit Wort gerade
	\\$\rightarrow$ Gerade Parität (d+1)tes Bit wird auf 1 gesetzt, wenn Anzahl der 1sen im d-Bit Wort ungerade
	\item Zweidimensionale Parität
	\\Zusätzliche Paritätsberechnung für jeweilige Bit-Position
	\item Internet-Prüfsummen
	\\ Sender interpretiert Nutzdaten als Folge von Ganzzahlen und berechnet die Summe
	\\\includegraphics[width=0.70\textwidth]{"graphics/prufsum"}
	\item Cyclic Redundancy Check (CRC)
	\\\includegraphics[width=0.80\textwidth]{"graphics/CRC"}
\end{itemize}
\section{Prinzipien der gesicherten Datenübertragung}
Grundprinzip der gesicherten Übertragung:
\begin{itemize}
	\item Prinzip der positiven Bestätigung (ACK+):
	\\erfolgreicher Erhalt wird mit einem "ACK+"-Paket bestätigt
	\\Nach Versand des Pakets wird auf das ACK+ gewartet
	\\Falls Wartezeitüberschritten, erfolgt Sendewiederholung
	\item Sendepuffer
	\\Datensegmente können verloren gesicherten
	\\Sender muss eine Kopie der Daten halten
	\item Sequenz- \& Bestätigungsnummern:
	\\Datensegmente können verdoppelt werden
	\item Problem:
	\\Je nach Ausbreitungsverzögerung sehr geringe effektive Übertragungsrate
\end{itemize}
Funktionsweise von Sliding-Window-Protokollen:
\\Bei ACK wird die Übertragungskapazität schlecht ausgenutzt
\\Jetzt: Sender schickt mehrere Frames, ohne auf ACKs zu warten
\\\includegraphics[width=0.70\textwidth]{"graphics/sliding-window"}
Varianten von Sliding-Window-Protokollen:
\begin{itemize}
	\item "Go-Back-n"-Strategie: RWS = 1
	\\Empfangspuffer des Empfängers kann genau ein Datensegement zwischen puffern
	\item "Selective Repeat"-Strategie: RWS $\leq$ 1
	\\Empfangspuffer des Empfängers kann mehrere Datenrahmen zwischen puffern
\end{itemize}
\section{Ethernet (IEEE 802.3)}
\includegraphics[width=0.70\textwidth]{"graphics/Ethernet-Standard"}
\subsection{Ethernet-Funktionsprinzip}
Alle Teilnehmer eines LANs teilen sich die Übertragungskapazität \textbf{"shared network"}
\\Alle Stationen "sehen" alle Daten-Rahmen im LANs
\\CSMA/CD: Medienzugriffsprotokoll für Ethernet
\\\tab CSMA/CD: Carrier Sense Multiple Access/Collision Detect
\\\tab Ablauf:
\\\tab[2cm] Sendewilige Station hört Leitung
\\\tab[2cm] Bei freier Leitung wird gesendet
\\\tab[2cm] Während der Sendung wird überwacht, ob Datenkollision auftritt
\\\tab[2cm] Bei Kollision: Sendung wird abgebrochen
\\Zeitlicher Ablauf bei einer Datenkollision:
\\\includegraphics[width=0.70\textwidth]{"graphics/CSMA"}
\\Der Konfliktparameter K:
\\\includegraphics[width=0.70\textwidth]{"graphics/Konfliktparameter"}
\subsection{MAC (Ethernet)-Adresse}
Länge: 6 Bytes bzw. 48 Bits
\\Broadcast-Adresse: Alle Bits der LAN-Adresse sind auf 1 gesetzt
\subsection{Funktion eines Ethernet-Adapters}
Der Ethernet-Adapter überprüft jeden gesendeten Rahmen (Hardware)
\\Wenn die Ziel-MAC-Adresse eines Rahmen = der lokalen MAC-Adresse des Adapters:
\\\tab Rahmen wird an das Betriebssystem weitergeleitet 
\\\tab Ausnahme: Promiscuous Mode oder MAC-Broadcast
\subsection{Netzwerkkomponenten}
\begin{itemize}
	\item Ethernet-Repeater:
	\begin{itemize}
		\item Zweck
		\\Längenbeschränkung aufgrund von Signaldämpfung aufheben
		\\Kopplung gleichartiger Netsegmente
		\item Funktion
		\\Arbeitet auf der Bitübertragungsschicht
		\\Verstärker wird zwischen zwei Segmente geschaltet
	\end{itemize}
	\item Ethernet-Hubs
 	\begin{itemize}
	 	\item Zweck ähnlich wie von Repeatern
		\\Repeater koppeln Segmente - Hubs bilden den zentralen Bus eines Segments
		\item Funktion
		\\Hubs arbeiten auf der Bitübertragungsschicht
		\\Das Gesamtnetz bildet eine Kollisionsdomäne
 	\end{itemize}
	\item "Repeater"-Regeln
	\\Anzahl der kaskadierbaren Hubs/Repeater ist begrenzt (max. 5 Segmente durch 4 Hubs/Repeater)
	\\\includegraphics[width=0.70\textwidth]{"graphics/repeater-regeln"}
	\item Ethernet-Bridges
	\begin{itemize}
		\item Kopplung zweier Ethernet-Segmente mit folgenden Eigenschaften
		\\Geschwindigkeitskonversion und Aufhebung der Repeater-Regeln
		\item Funktion
		\\Bridges sind Geräte der ISO/OSI-Schicht 2 (Bit-/Sicherungsschicht)
		\\Bridges entkoppeln Kollisionsdomänen
	\end{itemize}
	\item "Multiport"-Bridge
	\\Bridge mit mehr als zwei LAN-Schnittstellen
	\\enthält Informationen zur Filterung und Weiterleitung von Rahmen
	\item Selbstlernende Bridges
	\\Bridge "lernt" die Tabelleneinträge selbstständig
	\item Ethernet-Switches
	\\geringere Durchlaufverzögerung, mehr Ports und höherer Durchsatz als bei der Multiport-Bridge
	\item VLANs
	\\Switch verwaltet mehrere unabhängige Broadcast-Domänen
	\\Rechner können in unabhängige VLANs eingeteilt werden
\end{itemize}
\section{Wireless LAN (IEEE 802.11)}
Nutzt lizenzfreies 2,4GHz bzw. 5GHz Bandbreite
\\Erreichbare Datenraten
\begin{itemize}
	\item IEEE 802.11b \tab 11Mbps
 	\item IEEE 802.11g \tab 54Mbps
	\item IEEE 802.11a \tab 54Mbps $\rightarrow$ USA
	\item IEEE 802.11n \tab 600Mbps
	\item IEEE 802.11ac \tab 1700Mbps
\end{itemize}
Erreichbare Reichweiten: 30-50m im Gebäude, bis 1km außerhalb
\subsection{WLAN-Betriebsmoden}
\begin{itemize}
	\item Ad hoc Modus
	\\Direkter Verbindungsaufbau zwischen WLAN-Knoten
	\\Knoten müssen die gleiche Übertragungskanal-Nr. und SSID verwenden
	\item Infrastruktur-Modus
	\\WLAN-Clients kommunizieren über den Access Point (AP)
	\\Access Point wirkt wie eine Bridge zw. Funknetz und drahtgebundenem Netz
	\\über Broadcast werden Funknetz-Parameter verteilt
\end{itemize}
\subsection{Nutzbare Frequenzbänder für IEEE 802.11b/g}
13 überlappende Frequenzbänder (Europa)
\\\tab Standard-Kanal-Nummern: 1,6,11
\\Überlappungen führen zu Störungen und Bitrateneinbußen
\subsection{CSMA/CA: Medienzugriffsprotokoll für WLAN}
Kapitel 3 S.56-59
\subsection{Sicherheitsmechanismen für 802.11-Netze}
\begin{itemize}
	\item Alte Sicherheitsmechanismen:
 	\begin{itemize}
	 	\item (E)SSID, (Extended)ServiceSetIdentity: Kennung des Netzes
		\\WLAN-Client braucht die Kennung, um sich anzumelden
		\\wird oft durch Beacon-Frames vom AP selbst bekannt gemacht
		\item MAC-ACLs (Media Access Control-Access Lists)
		\\AP fürht List mit erlaubten Client-MAC-Adressierung
		\\MAC-Adresse können modifiziert werden
		\item WEP (Wired Equivalent Privacy)-Verschlüsselung
 	\end{itemize}
	\item Neue Sicherheitsmechanismen:
	\\Aktueller Sicherheitsstandard - WPA
\end{itemize}
\textbf{WEP-Verschlüsselung}
\\\includegraphics[width=0.70\textwidth]{"graphics/WEP"}
\\Der WEP-Schlüssel besteht aus zwei Teilen
\begin{itemize}
	\item [1.] "Geheimer" Benutzerschlüssel:
	\\Muss auf allen berechtigten Edngeräten eingetragen werden
	\\WEP64: enthält 40Bit-Benutzerschlüssel
	\item [2.] 24-Bit-Initialisierungsvektor (IV):
	\\Wird für "jedes" verschickte Datenpaket geändert
\end{itemize}
\textbf{Schwächen der WEP-Verschlüsselung}
\\IV ist zu kurz (nur 24 Bit)
\\\tab ca. alle 16 Mio Datenpakete wiederholt sich der IV
\\\tab IV wird im Klartext übertragen
\\Keine Schlüsselveraltung
\\\tab Speicherugn der geheimen Benutzerschlüssel in jedem Client
\\\tab Invalidierung eines Schlüssel erfolgt manuell
\\\textbf{Sicherheit von 802.11-Funktnetzen}
%%Kap 3 S69
%Kapitel 3 Seite 56
\end{document}